\documentclass[letterpaper]{article}
\usepackage[utf8]{inputenc}
\usepackage[spanish]{babel}
\usepackage{amssymb, amsmath}
\usepackage{graphicx}
\usepackage{lipsum}
\usepackage{dsfont}
\usepackage[margin=1.5cm,
vmargin={1.5cm,1.3cm},
includefoot]{geometry}
\usepackage{setspace}
\usepackage{subcaption}
\usepackage{tocloft}
\usepackage{upgreek}
\usepackage{amsthm}
\usepackage{graphicx}
\usepackage{paralist}
\usepackage{fancyhdr}
\usepackage{lmodern}
\usepackage{tcolorbox}
\usepackage{color}
\usepackage{tikz}
\tcbuselibrary{skins,breakable}
\pagestyle{fancy}

\renewcommand{\headrulewidth}{0.4pt}
\renewcommand{\footrulewidth}{0.4pt}

\providecommand{\abs}[1]{\left|#1\right|}
\providecommand{\norm}[1]{\left|\left|#1\right|\right|}														  
\newcommand{\tq}{ \quad \cdot  \backepsilon \cdot \quad }

\newcommand{\ld}{\lim\limits_{x \to 0^{+}}}

\newcommand{\li}{\lim\limits_{x \to 0^{-}}}

\newcommand{\la}{\lim\limits_{x \to a}}

\newcommand{\R}{\mathds{R}}

\newcommand{\Z}{\mathds{Z}}

\newcommand{\N}{\mathds{N}}

\renewcommand{\P}{\mathcal{P}}

\newcommand{\Po}{\mathds{P}_2(\mathds{R})}

\renewcommand{\*}{\cdot}

\makeatletter
\renewcommand*\env@matrix[1][*\c@MaxMatrixCols c]{%
	\hskip -\arraycolsep
	\let\@ifnextchar\new@ifnextchar
	\array{#1}}
\makeatother

\newtheorem{theorem}{Teorema}[section]
\theoremstyle{definition}
\newtheorem{definition}{Definición}

\begin{document}
\setlength{\unitlength}{1cm}
\thispagestyle{empty}
\begin{picture}(19,3)
\put(-0.5,1.2){\includegraphics[scale=.20]{unam1.png}}
\put(16,1){\includegraphics[scale=.29]{fciencias1.png}}
\end{picture}

\begin{center}
	\vspace{-114pt}
	\textbf{\large Álgebra Superior I}\\
	\textbf{ Semestre 2020-2}\\
	Prof. Alejandro Dorantes Aldama\\
	Ayud. Elmer Enrique Tovar Acosta \\
	Ayud. Alejandro Ríos Herrejón \\
	\textbf{Tarea II}
\rule{19cm}{0.2mm}
	\begin{center}
Kevin Ariel Merino Peña
	\end{center}
\rule{19cm}{0.2mm}
\end{center}
 %%%%%%%%%%%%%%%%%%%%%%%%%%%%%%%%%%%%%%%%%%%%
 %			|	EJERCICIO 3					%
 %			|	ARIEL						%	
 %%%%%%%%%%%%%%%%%%%%%%%%%%%%%%%%%%%%%%%%%%%%
\noindent3. Demuestre que toda función $ f: \R \to \R $ lineal 
\[ f(x) = ax +b \]
Es biyectiva siempre que $ a \neq 0 $\\

Sean $ f(x) $ y $ f(z)  \in \R$ entonces,
\begin{align*}
	f(x) &= f(z) &&\text{ Tomemos estas dos imágenes }\\
	ax +b &= az + b &&\text{ Por la regla de correspondencia de } f\\
	ax &= az &&\text{ Sumando el inverso aditivo de b en ambos miembros}\\
	x &= z &&\text{ porque } a \neq 0
\end{align*}
\begin{center}
	$ \therefore f $ es inyectiva
\end{center}
Luego, sea $ x \in \R $
\begin{align*}
	x &= r + b &&\text{ Esto puede ocurrir para todo número real, con }r, b \in \R \\
	x &= rn + b &&\text{ Además podemos ver a cualquier real como el producto de dos reales }n \in \R \\
\end{align*}
\begin{center}
	$ \therefore x \in Img(f)  \implies \R \subseteq Img(f)$\\
	La otra contención está dada por definición de la imagen de una función\\
	$ \therefore \quad Img(f) = \R  \implies f$ es suprayectiva.\\
	$ \therefore f $ es biyectiva.
\end{center}
Además, si $ g(x) = cx + d $ es otra función lineal, demuestre que \[f = g\] si y sólo si $a = c $, $ b = d $ si y sólo si \[ f(0) = g(0) \qquad \text{ y } \qquad f(1) = g(1) \]
$ \implies_1 $
\begin{align*}
	f(0) = g(0) \implies a(0) + b &= c(0) + d && \text{ Por regla de correspondecia de } f, g\\
	b &= d && \text{ Porque } a\* 0 = 0 \forall a \in \R 
\end{align*}
\begin{align*}
	f(1) = g(1) \implies a(1) + b &= c(1) + d && \text{ Por regla de correspondecia de } f, g\\
	 a + b &=c + d && \text{ Porque } x\* 1 = 0 \forall x \in \R \\
	 a + b &=c + b && \text{ Porque de lo anterior dedcimos que } b = d \\
	 a &=c && \text{ Sumando en ambos miembros el inverso aditivo de  } b  
\end{align*}
$ \impliedby_1 $
\begin{align*}
	a &= c &&\text{Por hipótesis}\\
	ax &= cx &&\text{multiplicando ambos miembros por } x\\
	ax + b &= cx + d&&\text{Pues por hipótesis } b = d\\
	a(1) + b &= c(1) + d&&\text{Sustituyendo } x = 1\\
	a(0) + b &= c(0) + d&&\text{Evalando en } x = 0
\end{align*}
De lo anterior tenemos que 
\[ a = c \land b = d \iff f(0) = g(0) \land f(1) = g(1) \]
$ \implies_2 $
\begin{align*}
	f = g \implies f (1) &= a(1) + b &&\text{Por regla de corespondencia de} f \\
	 &= c(1) + d &&\text{por la regla de correspondencia de }g \\
	 &= g(1) &&\text{pues suponemos que } f = g \\ \\
	f = g \implies f (0) &= a(0) + b &&\text{Por regla de corespondencia de} f \\
	 &= b &&\text{Por propiedades de los reales} \\
	 &= c (0) + d &&\text{por la regla de correspondencia de }g \\
	 &= g(0) &&\text{pues suponemos que } f = g 
\end{align*}
$ \impliedby_2 $ (Emplearemos la transitividad de $ \iff $)
\begin{align*}
	a = c \land b = d \implies f(x) &= a(x) + b &&\text{Por regla de correspondencia de } f\\
	\implies &= c(x) + d &&\text{Sutituyendo lo que estamos suponiendo }\\
	\implies &= g(x) &&\text{Por la regla de correspondencia de } g
\end{align*}
$$ \therefore \quad f = g \iff a = c \land b = d \iff f(0) = g(0) \land f(1) = g(1)  $$


%%%%%%%%%%%%%%%%%%%%%%%%%%%%%%%%%%%%%%%%%%%%%
%			|	EJERCICIO 8 				%
%			|	ARIEL						%	
%%%%%%%%%%%%%%%%%%%%%%%%%%%%%%%%%%%%%%%%%%%%%
\noindent8. Sea $ X = \{ f: \N \to \{1,0\} \mid f \text{ es función } \}  $. Dé una biyeccióń entre $ X $ y $ \P(\N) $\\


%%%%%%%%%%%%%%%%%%%%%%%%%%%%%%%%%%%%%%%%%%%%%
%			|	EJERCICIO 12				%
%			|	ARIEL						%	
%%%%%%%%%%%%%%%%%%%%%%%%%%%%%%%%%%%%%%%%%%%%%
\noindent12. Sea $ X $ un conjunto. Defina una relación $ R $ en $ \P(\N) $ como sigue
\[ (U,V) \in R \iff \abs{U} = \abs{V} \]
Demuestre que $ R $ es reflexiva, transitiva y simétrica. (emplearemos $ \sim $ para denotar que dos elementos están relacionados)\\
Sea $ X \in \P(\N)$, como $ \abs{X} = \abs{X} $, entonces
\[ X \sim X  \]
\begin{center}
	y como $ X $ es arbitrario, $\therefore \forall X \in \P(\N) \quad X \sim X  $\\
	$ \therefore \quad R $ es reflexiva
\end{center}
Sean $ X, Y, Z \in \P(\N) $
Supongamos $ X \sim Y $ y $ Y \sim Z $, lo anterior es $ \abs{X} = \abs{Y} $ y $ \abs{Y} = \abs{Z} $, como $ = $ es una relación binaria transitiva, entonces $ \abs{X} = \abs{Z} $ \textit{i.e. } $ X \sim Z $ 
\begin{center}
	y como $ X,Y, Z $ son arbitrarios, $\therefore \forall X, Y,Z \in \P(\N) \quad X \sim Y \land Y \sim  Z \implies X \sim Z  $\\
	$ \therefore \quad R $ es transitiva.
\end{center}
Sean $ X,Y \in \P(\N)$
supongamos $ X \sim Y $ entonces $ \abs{X} = \abs{Y} $ y como $ = $ es una relación simétrica, entonces $ \abs{Y} = \abs{X} $ \textit{i.e. }
$ Y \sim X $.
\begin{center}
	y como $ X,Y $ son arbitrarios, $\therefore \forall X, Y \in \P(\N) \quad X \sim Y  \implies Y \sim X  $\\
	$ \therefore \quad R $ es simétrica.\\
	(entonces es relación de equivalencia)
\end{center}
%%%%%%%%%%%%%%%%%%%%%%%%%%%%%%%%%%%%%%%%%%%%%
%			|	EJERCICIO 20				%
%			|	Ariel						%	
%%%%%%%%%%%%%%%%%%%%%%%%%%%%%%%%%%%%%%%%%%%%%
\noindent20. Demuestre por inducción:
$$1+r+r^2+r^3+...+r^n=\frac{1-r^{n+1}}{1-r}$$
\begin{flushright}
	\textbf{Caso base} 
\end{flushright}
$ n=1 $, El primer sumando del primer miembro es 1 el último es $r^1$, así tenemos\\
$$1+r$$ 
El segundo miembro de la igualdad, queda $$\frac{1-r^2}{1-r}=\frac{(1+r)(1-r)}{1-r}$$
entonces $$1+r=1+r$$ 
\begin{center}
	$ \therefore $  Se cumple para $n=1$
\end{center}

\begin{flushright}
	\textbf{Hipótesis de inducción}
\end{flushright}
Suponemos que la proposición es válida para $n \in \mathbb{N}.$\\
$$1+r+r^2+r^3+...+r^n=\frac{1-r^{n+1}}{1-r}$$
\begin{flushright}
\textbf{Paso inductivo}
\end{flushright}
P.d. Se cumple para $ n=n+1 $
\begin{align*}
	1+r+r^2+r^3+...+r^n+r^{n+1}&=\frac{1-r^{n+1}}{1-r}+r^{n+1}...&& \text{Por hipótesis de inducción}\\
	&=\frac{1-r^{n+1}+r^{n+1}-r^{n+2}}{1-r} &&\text{Operando suma de fracciones}\\
	&=\frac{1-r^{(n+1)+1}}{1-r} && \text{Reduciendo términos}\\
\end{align*}
\begin{center}
	$\therefore$ Se  cumple  $\forall{n} \in \mathbb{N}$\\
\end{center}



%%%%%%%%%%%%%%%%%%%%%%%%%%%%%%%%%%%%%%%%%%%%%
%			|	EJERCICIO 24				%
%			|	ARIEL						%	
%%%%%%%%%%%%%%%%%%%%%%%%%%%%%%%%%%%%%%%%%%%%%
\noindent Dado $ A \subseteq \N $, decimos que $ x \in A $ es el máximo de $ A $ si para todo $ y \in A $, se cumple $ y \leq x $. Demuestree que todo subconjunto finito de números naturales tiene máximo usando inducción sobre la cardinalidad del conjunto.

\noindent24. Usando el ejercicio anterior, demuestre que $ \N $ no es finito.\\

Procederemos por contradicción, suponiendo que $ \N $ es finito, entonces por el ejercicio anterior podemos hallar un máximo. \\
Sea $ x  $ el elemento máximo, entonces se cumple que $$ \forall y \in \N , \quad y \leq x $$ Ahora veamos al sucesor de $ x $, pues por un axioma de \textit{Peano} sabemos todos los elementos de $ \N $ tienen un sucesor, así que $$ \exists x + 1 \in \N $$luego $$ x \leq x+1 !$$ entonces $ x $ ya no sería elemento máximo, esta contradicción vino de suponer que $ \N $ es finito
\begin{center}
	$ \therefore \quad \N$  tiene una cantidad infinita de elementos.
\end{center}
\end{document}