\documentclass[letterpaper]{article}
\usepackage[utf8]{inputenc}
\usepackage[spanish]{babel}
\usepackage{amssymb, amsmath}
\usepackage{graphicx}
\usepackage{lipsum}
\usepackage{dsfont}
\usepackage[margin=1.5cm,
vmargin={1.5cm,1.3cm},
includefoot]{geometry}
\usepackage{setspace}
\usepackage{subcaption}
\usepackage{tocloft}
\usepackage{upgreek}
\usepackage{amsthm}
\usepackage{graphicx}
\usepackage{paralist}
\usepackage{fancyhdr}
\usepackage{lmodern}
\usepackage{tcolorbox}
\usepackage{color}
\usepackage{tikz}
\tcbuselibrary{skins,breakable}
\pagestyle{fancy}

\renewcommand{\headrulewidth}{0.4pt}
\renewcommand{\footrulewidth}{0.4pt}

\providecommand{\abs}[1]{\left|#1\right|}
\providecommand{\norm}[1]{\left|\left|#1\right|\right|}														  
\newcommand{\tq}{ \quad \cdot  \backepsilon \cdot \quad }

\newcommand{\ld}{\lim\limits_{x \to 0^{+}}}

\newcommand{\li}{\lim\limits_{x \to 0^{-}}}

\newcommand{\la}{\lim\limits_{x \to a}}

\newcommand{\R}{\mathds{R}}

\newcommand{\Z}{\mathds{Z}}

\newcommand{\N}{\mathds{N}}

\renewcommand{\P}{\mathcal{P}}

\newcommand{\Po}{\mathds{P}_2(\mathds{R})}

\renewcommand{\*}{\cdot}

\makeatletter
\renewcommand*\env@matrix[1][*\c@MaxMatrixCols c]{%
	\hskip -\arraycolsep
	\let\@ifnextchar\new@ifnextchar
	\array{#1}}
\makeatother

\newtheorem{theorem}{Teorema}[section]
\theoremstyle{definition}
\newtheorem{definition}{Definición}

\begin{document}
\setlength{\unitlength}{1cm}
\thispagestyle{empty}
\begin{picture}(19,3)
\put(-0.5,1.2){\includegraphics[scale=.20]{unam1.png}}
\put(16,1){\includegraphics[scale=.29]{fciencias1.png}}
\end{picture}

\begin{center}
	\vspace{-114pt}
	\textbf{\large Álgebra Superior I}\\
	\textbf{ Semestre 2020-2}\\
	Prof. Alejandro Dorantes Aldama\\
	Ayud. Elmer Enrique Tovar Acosta \\
	Ayud. Alejandro Ríos Herrejón \\
	\textbf{Tarea II}
\rule{19cm}{0.3mm}
	Carlos Andrade Hernández \hspace{2cm} Berenice SanJuan Hernández \hspace{2cm} Carlos Uriel Sánchez Martínez \\
	\begin{center}
	Edgar Samuel Palacios Crispín \hspace{2cm} Kevin Ariel Merino Peña
	\end{center}
	\vspace{-14pt}
\rule{19cm}{0.3mm}
\end{center}
 %%%%%%%%%%%%%%%%%%%%%%%%%%%%%%%%%%%%%%%%%%%%
 %			|	EJERCICIO 3					%
 %			|	ARIEL						%	
 %%%%%%%%%%%%%%%%%%%%%%%%%%%%%%%%%%%%%%%%%%%%
\noindent3. Demuestre que toda función $ f: \R \to \R $ lineal 
\[ f(x) = ax +b \]
Es biyectiva siempre que $ a \neq 0 $\\

Además, si $ g(x) = cx + d $ es otra función lineal, demuestre que \[f = g\] si y solamente si $a = c $, $ b = d $ si y solamente si \[ f(0) = g(0) \qquad \text{ y } \qquad f(1) = g(1) \]

%%%%%%%%%%%%%%%%%%%%%%%%%%%%%%%%%%%%%%%%%%%%%
%			|	EJERCICIO 7					%
%			|	ARIEL						%	
%%%%%%%%%%%%%%%%%%%%%%%%%%%%%%%%%%%%%%%%%%%%%
\noindent7. Sea $ X $ un conjunto. Demuestre que no hay una biyección entre $ X $ y $ \P(X) $ terminando el siguiente argumento:
\begin{center}
\parbox[b][3cm][c]{0.7\textwidth}{Supongamos que existe dicha biyección, digamos $ f: X \to \P(X) $. Dado $ x \in X $, dado que $ f(x) \subseteq X $ tiene sentido preguntarnos ¿$ x \in f(x) $? denote por $ A $ al conjunto de los puntos que no cumplen esto, es decir,
\[ A = \{ x \in X | x \notin f(x) \} \]
Ahora, como $ A \subseteq X $ y $ f $ es biyectiva $ y = f^{-1}(A) \in X $. Entonces ¿$ y \in A $ o $ y \notin A $ ? 
}
\end{center}



%%%%%%%%%%%%%%%%%%%%%%%%%%%%%%%%%%%%%%%%%%%%%
%			|	EJERCICIO 11				%
%			|	ARIEL						%	
%%%%%%%%%%%%%%%%%%%%%%%%%%%%%%%%%%%%%%%%%%%%%
\noindent11.Demuestre que no ha una biyección entre $ \N $ y $ \R $. \textit{Sugerencia: Los ejercicios anteriores pueden servir}

%%%%%%%%%%%%%%%%%%%%%%%%%%%%%%%%%%%%%%%%%%%%%
%			|	EJERCICIO 15				%
%			|	ARIEL						%	
%%%%%%%%%%%%%%%%%%%%%%%%%%%%%%%%%%%%%%%%%%%%%
\noindent15.Diga si las siguientes son particiones de los conjuntos dados y si lo son, diga cuáles son las relaciones de equivalencia inducidad, justificando bien sus respuestas
\begin{itemize}
	\item  En $ \Z $, sea $ P = \{ \{ n \in \Z : \exists m \in \Z(n = 3m) \}, \{ n \in \Z : \exists m \in \Z(n = 3m+1 \}, \{ n \in \Z : \exists m \in \Z(n = 3m + 2)  \} \}$
	\item En $  \Z $, sea $ P = \{ \{ n \in \Z :n \leq 0 \}, \{ n \in \Z: n > 1 \} \}  $
	\item En el conjunto $ A= \{1,2,3,4,5,6 \} $, sea $ P = \{ \{ 1,2\},\{3,4,5,6 \},\{6 \} \} $
	\item En el conjunto $ A= \{1,2,3,4,5,6 \} $, sea $ P = \{ \{ 1,2\},\{3,4,6 \},\{5 \} \} $
	\item En $ \R $, sea $ P = \{ \{ x \in \R: x \leq 0 \land x \text{ es irracional } \},\{x \in \R: x \leq 0 \land x \text{ es racional} \},\{x \in \R: x>0 \},\{x \in \R: x \text{ es irracional } \land x \text{ es racional } \}  \} $ 
	\item En $ \R $ sea $ P = \{  \{ x \in \R: x \leq 0 \land x \text{ es irracional} \},\{ x \in \R : x \leq0 \land x \text{ es racional} \},\{ x \in \R: x > 0 \}  \} $
\end{itemize}


%%%%%%%%%%%%%%%%%%%%%%%%%%%%%%%%%%%%%%%%%%%%%
%			|	EJERCICIO 16				%
%			|	BERE						%	
%%%%%%%%%%%%%%%%%%%%%%%%%%%%%%%%%%%%%%%%%%%%%
\noindent16. Demuestre por inducción:
$$1+2+3+....+ n = \frac{n(n+1)}{2}$$
Caso base: n=1
$$1 = \frac{1(1+1)}{2} = \frac{2}{2} = 1$$
Paso Inductivo: Suponemos que la proposición es cierta para $n=k \in \mathbb{N}.$\\
H.I:
$$1+2+3+....+ k = \frac{k(k+1)}{2}$$
P.D que la proposición es cierta para n=k+1
$$1+2+3+....+ k + 1 = (1+2+3+...+k) + (k+1)$$
\begin{align*}
	&=\frac{k(k+1)}{2}+k+1\\
	&=\frac{k(k+1)+2(k+1)}{2}\\
	&=\frac{(k+1)(k+2)}{2}\\
	&=\frac{k+1[(k+1)+1]}{2}
\end{align*}

\begin{center}
$\therefore$ Se cumple  $\forall{n} \in \mathbb{N}$
\end{center}
%%%%%%%%%%%%%%%%%%%%%%%%%%%%%%%%%%%%%%%%%%%%%
%			|	EJERCICIO 17				%
%			|	BERE						%	
%%%%%%%%%%%%%%%%%%%%%%%%%%%%%%%%%%%%%%%%%%%%%
\noindent17. Demuestre por inducción:
$$1^2+2^2+....+n^2=\frac{n(n+1)(2n+1)}{6}$$
Caso base: n=1
$$1^2=\frac{1(1+1)(2(1)+1)}{6}=\frac{1(2)(3)}{6}=\frac{6}{6}=1$$
Paso Inductivo: Suponemos que la fórmula es válida para $n=k \in \mathbb{N}.$\\
H.I:$$1^2+2^2+...+k^2=\frac{k(k+1)(2k+1)}{6}$$
P.D. que la proposición es cierta para n=k+1
$$ 1^2+2^2+...+k+1^2=(1^2+2^2+...+k^2)+(k+1)^2  $$
\begin{align*}
	&=\frac{k(k+1)(2k+1)}{6}+(k+1)^2\\
	&=\frac{k(k+1)(2k+1)+6(k+1)^2}{6}\\
	&=\frac{(k+1)[k(2k+1)+6(k+1)]}{6}\\
	&=\frac{(k+1)(2k^2+7k+6)}{6}\\
	&=\frac{(k+1)(k+2)(2k+3)}{6}\\
	&=\frac{(k+1)[(k+1)+1][2(k+1)+1]}{6}
\end{align*}
\begin{center}
	$\therefore$ Se  cumple $\forall{n} \in \mathbb{N}$	
\end{center}

%%%%%%%%%%%%%%%%%%%%%%%%%%%%%%%%%%%%%%%%%%%%%
%			|	EJERCICIO 18				%
%			|	BERE						%	
%%%%%%%%%%%%%%%%%%%%%%%%%%%%%%%%%%%%%%%%%%%%%
\noindent18. Encuentre una fórmula para la siguiente suma, luego 
demuestre dicha fórmula por inducción:
$$1+3+5+7+....+2n-1$$
Fórmula: La suma de n primeros números impares consecutivos es $n^2$, es decir:\\
$$1+3+5+7+....+2n-1=n^2$$
Caso base: (Abarcaremos un par de casos)\\
a) n=1\\
$2(1)-1=1^2$ (Primer número de impar)\\
$1=1$\\
b) n=2\\
$1+2(2)-1=2^2$ (Se suma el primer número consecutivo, por eso el resultado es 4)\\
$4=4$\\
Paso Inductivo: Suponemos que la fórmula es válida para $n=k \in \mathbb{N}.$\\
H.I 
$$1+3+5+7+....+2k-1=k^2$$
P.D. Suponemos que la fórmula es válida para $n=k \in \mathbb{N}.$\\
\begin{align*}
	1+3+5+7+....+2(k-1)+(2(k+1)-1)&=(k+1)^2\\
	1+3+5+7+....+2(k-1)+(2(k+1)-1)&=k^2+(2(k+1)-1)\\
	&=k^2+2k+2-1\\
	&=k^2+2k+1\\
	&=(k+1)^2\\
\end{align*}
\begin{center}
	$\therefore$ Se cumple $\forall{n} \in \mathbb{N}$\\
\end{center}

%%%%%%%%%%%%%%%%%%%%%%%%%%%%%%%%%%%%%%%%%%%%%
%			|	EJERCICIO 19				%
%			|	BERE						%	
%%%%%%%%%%%%%%%%%%%%%%%%%%%%%%%%%%%%%%%%%%%%%
\noindent19.	Mismas instrucciones que antes para:
$$1^2+3^2+5^2+....+(2n-1)^2$$
Fórmula:
$$1^2+3^2+5^2+....+(2n-1)^2 = \frac{n(2n-1)(2n+1)}{3}$$
Caso base: n=1
\begin{align*}
	(2(1)-1)^2&=\frac{1(2(1)-1)(2(1)+1)}{3}\\
	1&=\frac{1(1)(3)}{3}\\
	1&=1
\end{align*}
Paso Inductivo: Suponemos que la fórmula es válida para $n=k \in \mathbb{N}.$\\
H.I
$$1^2+3^2+5^2+....+(2k-1)^2 = \frac{k(2k-1)(2k+1)}{3}$$
P.D que la proposición es cierta para n=k+1
$$1^2+3^2+5^2+....+(2k-1)^2+(2k+1)^2= \frac{k(2k-1)(2k+1)}{3}+(2k+1)^2$$
Podemos ver que hay un factor común en ambos términos $((2k+1) y (2k+1)^2)$, así que lo extraemos
\begin{align*}
	&=(2k+1)[\frac{k(2k-1)}{3}+2k+1]\\
	&=(2k+1)[\frac{k(2k-1)+3(2k+1)}{3}]\\
	&=(2k+1)[\frac{2k^2-k+6k+3}{3}]\\
	&=(2k+1)[\frac{2k^2+5k+3}{3}]\\
\end{align*}
Factorizamos el trinomio
$$=\frac{(2k+1)(2k+3)(k+1)}{3}$$
\begin{center}
	$\therefore$ Se cumple $\forall{n} \in \mathbb{N}$\\
\end{center}

%%%%%%%%%%%%%%%%%%%%%%%%%%%%%%%%%%%%%%%%%%%%%
%			|	EJERCICIO 20				%
%			|	BERE						%	
%%%%%%%%%%%%%%%%%%%%%%%%%%%%%%%%%%%%%%%%%%%%%
\noindent20. Demuestre por inducción:
$$1+r+r^2+r^3+...+r^n=\frac{1-r^{n+1}}{1-r}$$
Caso base: n=1\\
El primer sumando del primer miembro es 1 el último es $r^1$, así tenemos\\
$$1+r$$ 
El segundo miembro queda $$\frac{1-r^2}{1-r}=\frac{(1+r)(1-r)}{1-r}$$
entonces $$1+r=1+r$$ 
Se cumple para $n=1$\\

Paso Inductivo: Suponemos que la fórmula es válida para $n=k \in \mathbb{N}.$\\
H.I
$$1+r+r^2+r^3+...+r^k=\frac{1-r^{k+1}}{1-r}$$
P.D. que la proposición es cierta para n=k+1
\begin{align*}
	1+r+r^2+r^3+...+r^k+r^{k+1}&=\frac{1-r^{k+1}}{1-r}+r^{k+1}...Por H.I\\
	&=\frac{1-r^{k+1}+r^{k+1}-r^{k+2}}{1-r}\\
	&=\frac{1-r^{(k+1)+1}}{1-r}\\
\end{align*}
\begin{center}
	$\therefore$ Se  cumple  $\forall{n} \in \mathbb{N}$\\
\end{center}

\noindent21.Pruebe que:


%%%%%%%%%%%%%%%%%%%%%%%%%%%%%%%%%%%%%%%%%%%%%
%			|	EJERCICIO 23				%
%			|	ARIEL						%	
%%%%%%%%%%%%%%%%%%%%%%%%%%%%%%%%%%%%%%%%%%%%%
\noindent23. Dado $ A \subseteq \N $, decimos que $ x \in A $ es el máximo de $ A $ si para todo $ y \in A $, se cumple $ y \leq x $. Demuestree que todo subconjunto finito de números naturales tiene máximo usando inducción sobre la cardinalidad del conjunto.

\end{document}