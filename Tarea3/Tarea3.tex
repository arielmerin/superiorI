\documentclass[letterpaper]{article}
\usepackage[utf8]{inputenc}
\usepackage[spanish]{babel}
\usepackage{amssymb, amsmath}
\usepackage{graphicx}
\usepackage{lipsum}
\usepackage{dsfont}
\usepackage[margin=1.5cm,
vmargin={1.5cm,1.3cm},
includefoot]{geometry}
\usepackage{setspace}
\usepackage{subcaption}
\usepackage{tocloft}
\usepackage{upgreek}
\usepackage{amsthm}
\usepackage{graphicx}
\usepackage{paralist}
\usepackage{fancyhdr}
\usepackage{lmodern}
\usepackage{tcolorbox}
\usepackage{color}
\usepackage{tikz}
\tcbuselibrary{skins,breakable}
\pagestyle{fancy}

\renewcommand{\headrulewidth}{0.4pt}
\renewcommand{\footrulewidth}{0.4pt}

\providecommand{\abs}[1]{\left|#1\right|}
\providecommand{\norm}[1]{\left|\left|#1\right|\right|}														  
\newcommand{\tq}{ \quad \cdot  \backepsilon \cdot \quad }

\newcommand{\ld}{\lim\limits_{x \to 0^{+}}}

\newcommand{\li}{\lim\limits_{x \to 0^{-}}}

\newcommand{\la}{\lim\limits_{x \to a}}

\newcommand{\R}{\mathds{R}}

\newcommand{\Z}{\mathds{Z}}

\newcommand{\N}{\mathds{N}}

\renewcommand{\P}{\mathcal{P}}

\newcommand{\Po}{\mathds{P}_2(\mathds{R})}

\renewcommand{\*}{\cdot}

\makeatletter
\renewcommand*\env@matrix[1][*\c@MaxMatrixCols c]{%
	\hskip -\arraycolsep
	\let\@ifnextchar\new@ifnextchar
	\array{#1}}
\makeatother

\newtheorem{theorem}{Teorema}[section]
\theoremstyle{definition}
\newtheorem{definition}{Definición}

\begin{document}
\setlength{\unitlength}{1cm}
\thispagestyle{empty}
\begin{picture}(19,3)
\put(-0.5,1.2){\includegraphics[scale=.20]{unam1.png}}
\put(16,1){\includegraphics[scale=.29]{fciencias1.png}}
\end{picture}

\begin{center}
	\vspace{-114pt}
	\textbf{\large Álgebra Superior I}\\
	\textbf{ Semestre 2020-2}\\
	Prof. Alejandro Dorantes Aldama\\
	Ayud. Elmer Enrique Tovar Acosta \\
	Ayud. Alejandro Ríos Herrejón \\
	\textbf{Tarea III}
\rule{19cm}{0.2mm}
	\begin{center}
Kevin Ariel Merino Peña\footnote[2]{Número de cuenta: 317031326}
	\end{center}
\rule{19cm}{0.2mm}
\end{center}
 %%%%%%%%%%%%%%%%%%%%%%%%%%%%%%%%%%%%%%%%%%%%
 %			|	EJERCICIO 5					%
 %			|	ARIEL						%	
 %%%%%%%%%%%%%%%%%%%%%%%%%%%%%%%%%%%%%%%%%%%%
\noindent5.Enliste todas las ordenaciones de las letras $ a, b, c, d $ tomadas de tres en tres.\\

 %%%%%%%%%%%%%%%%%%%%%%%%%%%%%%%%%%%%%%%%%%%%
%			|	EJERCICIO 9					%
%			|	ARIEL						%	
%%%%%%%%%%%%%%%%%%%%%%%%%%%%%%%%%%%%%%%%%%%%
\noindent9.Escríbanse todas las permutaciones de los dıgitos $ 1, 2, 3, 4 $\\


 %%%%%%%%%%%%%%%%%%%%%%%%%%%%%%%%%%%%%%%%%%%%
%			|	EJERCICIO 15				%
%			|	ARIEL						%	
%%%%%%%%%%%%%%%%%%%%%%%%%%%%%%%%%%%%%%%%%%%%
\noindent15.Un juego de dominó consta de 28 fichas y una mano consta de 7 fichas.
¿De cuántas formas se puede seleccionar una mano?\\


 %%%%%%%%%%%%%%%%%%%%%%%%%%%%%%%%%%%%%%%%%%%%
%			|	EJERCICIO 18				%
%			|	ARIEL						%	
%%%%%%%%%%%%%%%%%%%%%%%%%%%%%%%%%%%%%%%%%%%%
\noindent18.¿Cuántas manos de póker hay que tengan exactamante una tercia y que no sea un full? Justifique su respuesta.\\


 %%%%%%%%%%%%%%%%%%%%%%%%%%%%%%%%%%%%%%%%%%%%
%			|	EJERCICIO 23				%
%			|	ARIEL						%	
%%%%%%%%%%%%%%%%%%%%%%%%%%%%%%%%%%%%%%%%%%%%
\noindent23.¿De cuántas formas diferentes es posible ordenar los símbolos $ s,a,r,a,s,e,r,a $?\\


 %%%%%%%%%%%%%%%%%%%%%%%%%%%%%%%%%%%%%%%%%%%%
%			|	EJERCICIO 30				%
%			|	ARIEL						%	
%%%%%%%%%%%%%%%%%%%%%%%%%%%%%%%%%%%%%%%%%%%%
\noindent30.¿Cuántas ordenaciones de las letras de \textit{PRINCIPIO} no tienen I consecutivas?
\end{document}